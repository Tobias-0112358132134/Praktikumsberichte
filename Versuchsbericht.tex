\documentclass[a4paper,12pt]{article}
\usepackage[T1]{fontenc}
\usepackage[utf8]{inputenc}
\usepackage{graphicx}
\usepackage{amsmath}
\usepackage{amsfonts}
\usepackage{amssymb}
\usepackage{booktabs}
\usepackage{float}
\usepackage{geometry}
\usepackage[german]{babel}
\usepackage{enumitem}
\usepackage{parskip}
\usepackage{underscore}
\usepackage{hyperref}
\usepackage{icomma}
% \usepackage{multicol}

\geometry{a4paper, left=25mm, right=25mm, top=20mm, bottom=20mm}

%%%%%%%%%%%%%%% Titelblatt %%%%%%%%%%%%%%%
\begin{document}
\begin{titlepage}
    \centering
    \includegraphics[scale = 0.03]{bilder/JKU_Logo.png}\\[1.0 cm]	% JKU Logo
    \textsc{\Large Einführungspraktikum Physik}\\[0.5 cm]	        % LVA Name
    \textsc{\large 3. Versuch}\\[0.5 cm]				            % Versuch Nummer    // [x] TODO: Versuch Nummer anpassen
    \rule{\linewidth}{0.4 mm} \\[0.4 cm]
    { \huge \bfseries Brennweite}\\                                 % Versuch Name      // [x] TODO: Versuch Name anpassen
    \rule{\linewidth}{0.4 mm} \\[1.5 cm]
    \begin{minipage}{0.8\textwidth}
        \begin{flushleft} \large
            \emph{Autoren:}\\
            Eva Brandstätter (k12406599)\\
            Tobias Mittermair (k12412801)\\
            \vspace{1cm}
            \emph{Gruppe:}\\
            Freitag Vormittag\\
            \vspace{1cm}
            \emph{Betreuer:}\\
            Gerald Gmachmeir
        \end{flushleft}
        \begin{flushright} \large
            \vspace{8cm}
            \emph{Abgabe:} \\
            \today
        \end{flushright}
    \end{minipage}~    
\end{titlepage}

%%%%%%%%%%%%%%% Inhaltsverzeichnis %%%%%%%%%%%%%%%
\tableofcontents
\newpage


%%%%%%%%%%%%%%% Inhalt %%%%%%%%%%%%%%%
\section{Einleitung}
%//[ ] TODO: Einleitung
% Was soll gemessen werden? (Ziel / Motivation / Hypothese / erwartetes Ergebnis)

In diesem Versuch soll die Brennweite und die zugehörige Unsicherheit einer Sammellinse durch 
das Bessel-Verfahren bestimmt werden.

\section{Grundlagen}
%//[ ] TODO: Grundlagen
% (kurz!) Was muss ich über die zu messende Größe wissen?

% Größe s beschreiben (Bezug auf optische bank)

Das Bessel-Verfahren leitet sich aus der Annäherung an achsennahe Parallelstrahlen und einer dünnen Sammellinse ab.
Diese Umstände sind annähernd erfüllt.
Der genaue Ablauf des Verfahrens kann der Versuchsdurchführung \ref{sec:Durchfuehrung} entnommen werden.

Die Formel zur Berechnung der Brennweite $f$ mit dem Bessel-Verfahren lautet:

\begin{equation}
    \label{eq:BesselBrennweite}
    f = \frac{s^2 - e^2}{4s}
\end{equation}

wobei $s$ der Abstand zwischen Gegenstand und Schirm und $e$ der Abstand zwischen den beiden
Positionen ($e_1$, $e_2$), an denen das Bild scharf ist, ist. Die Positionen $e_1$ und $e_2$
sind dabei symmetrisch zum Mittelpunkt von $s$.

Damit es tatsächlich zwei Positionen gibt, an denen ein scharfes Bild entsteht, muss der Gegenstand
in beiden Positionen weiter als $f$ von der Linse entfernt sein. Daraus folgt die Bedingung $s>4f$.

Als Anhaltspunkt für die Brennweite wird eine zweite Methode angewandt, die ein schnelles,
aber ungenaues Ergebnis liefert: es wird ausgenutzt, dass sich Parallelstrahlen annähernd in einem Punkt schneiden, wenn sie durch eine
Sammellinse fokussiert werden. Dieser Punkt ist der Brennpunkt der Linse und befindet sich in der Entfernung $f$
von der Linse.

\begin{equation}
    \label{eq:BesselAbstand}
    e = e_2 - e_1
\end{equation}

Da bei der Ermittlung der Position der Linse für eine Scharfstellung die Schärfe nur schwer (und nur subjektiv)
beurteilt werden kann, wird je ein Intervall $[e_{i_\mathrm{links}};e_{i_\mathrm{rechts}}]$ für $e_1$ und
$e_2$ bestimmt, in dem das Bild gerade noch scharf ist. Daraus wird $e_{i_\mathrm{Mitte}}$ folgendermaßen berechnet.

\begin{equation}
    \label{eq:BesselAbstandIntervall}
    e_{i_\mathrm{Mitte}} = \frac{e_{i_\mathrm{rechts}} + e_{i_\mathrm{links}}}{2}
\end{equation}

Da die Größen $s$, $e_{i_\mathrm{rechts}}$ und $e_{i_\mathrm{links}}$ gemessen wurden, sind sie mit einer
Messunsicherheit behaftet. 

%//[ ] in Grundlagen nur Existenz von Unsicherheiten erwähnen, nicht wie sie berechnet werden? (-> Auswertung?)

Da aus $e_{i_\mathrm{rechts}}$ und $e_{i_\mathrm{links}}$, die ein Vertrauensintervall beschreiben, aber eine
deutlich einflussreichere Unsicherheit für $e_i$ (um $e_{i_\mathrm{Mitte}}$) ermittelt werden kann, wird die Unsicherheit
für $e_i$ aus der Intervalllänge abgeleitet.
Da anschließend $e$ aus $e_1$ und $e_2$ berechnet wird, entspricht die Unsicherheit von $e$ der Kombination der
Unsicherheiten von $e_1$ und $e_2$.
Da zwischen $e_1$ und $e_2$ der geometrische Zusammenhang besteht, dass die Positionen symmetrisch zum Mittelpunkt
von $s$ liegen, wird von korrelierten Größen ausgegangen und die Unsicherheiten werden addiert:
%//[ ] -> nach diesem Argument, sind aber auch e und s korreliert,
%         also müsste man dann auch die normale Fehlerfortpflanzung anwenden?

% ---|--*e1*--|----------|--*e2*--|---

% e1 3sigma -> sigm_e1 0,2
% e2 3sigma -> sigm_e2 0,2

% e? max (e2+sigm_e2)-(e1-sigm_e1) = e2+sigm_e2-e1+sigm_e1 = e2-e1 + sigm_e2+sigm_e1 = e + (sigm_e2+sigm_e1) =: e + sigm_e
%    min (e2-sigm_e2)-(e1+sigm_e1) = e2-sigm_e2-e1-sigm_e1 = e2-e1 - sigm_e2-sigm_e1 = e - (sigm_e2+sigm_e1) =: e - sigm_e


\begin{equation}
    \label{eq:UnkorrUnsFortpfl}
    u_e=u_{e_1}+u_{e_2}
\end{equation}

%//[ ] wirklich Gauß?
Für die Brennweite wird die (Gauß'sche) Fortpflanzung von Unsicherheiten angewendet,
um daraus die Unsicherheit der Brennweite $u_f$ zu berechnen. Diese lautet wie folgt:

\begin{equation}
    \label{eq:GaussUnsFortpfl}
    u_f = \sqrt{\left(\frac{\partial f}{\partial s}\right)^2 \cdot u_s^2 + \left(\frac{\partial f}{\partial e}\right)^2 \cdot u_e^2}
\end{equation}

die partiellen Ableitungen lauten:

\begin{equation}
    \frac{\partial f}{\partial s} = \frac{s^2+e^2}{4s^2}
\end{equation}

\begin{equation}
    \frac{\partial f}{\partial e} = -\frac{e}{2s}
\end{equation}

daraus ergibt sich eingesetzt in die Gleichung \ref{eq:GaussUnsFortpfl}

\begin{equation}
    \label{eq:BesselBrennwUnsFortpfl}
    u_f(s,e) = \sqrt{\left(\frac{s^2+e^2}{4s^2}\right)^2 \cdot u_s^2 + \left(-\frac{e}{2s}\right)^2 \cdot u_e^2}
\end{equation}

Nun können die, aus den verschiedenen $s_i$ ermittelten, $f_i$ per gewichtetem Mittelwert zu einem
mittleren $\bar{f_g}$ zusammengefasst werden. Die Gewichte $g_i$ entsprechen dabei den inversen
Varianzen.

\begin{equation}
    \label{eq:Gewicht}
    g_i = \frac{1}{u_{f_i}^2}
\end{equation}

\begin{equation}
    \label{eq:Mittelwert}
    \bar{f_g} = \frac{\sum_{i=1}^{N} g_i \cdot f_i}{\sum_{i=1}^{N} g_i}
\end{equation}

Dieser gewichtete Mittelwert verfügt auch über eine gewichtete Unsicherheit $u_{\bar{f_g}}$.

\begin{equation}
    \label{eq:GewichtUnsicherheit}
    u_{\bar{f_g}} = \sqrt{\frac{1}{\sum_{i=1}^{N} g_i}}
\end{equation}

\section{Versuchsbeschreibung}
\subsection{Versuchsaufbau}
%//[x] TODO: Versuchsaufbau
% Wie sieht der Versuchsaufbau aus? (Skizze, Anleitung, Geräte, …)
% auf Abbildungen Bezug nehmen

Der Versuch wurde am 13.12.2024 im Praktikumsraum P122 an der JKU in Linz zwischen 11:15 und
13:15 durchgeführt.

Für diesen Versuch wurden eine Sammellinse, ein Lineal, eine optische Bank (mit Reitern, LED
mit Kondensorlinse, Geodreieck als Objekt und Schirm) bereitgestellt.
Weiters stand ein Laptop zur Führung des Laborprotokolls und zur Dokumentation der Werte bereit.
In der nachstehenden Abbildung \ref{Abb:Versuchsaufbau1} ist der Aufbau des Versuches dargestellt.
Dabei ist zu erkennen, dass von links weg eine LED mit Kondensorlinse, danach der Objektträger
mit Geodreieck und darauffolgend eine Linse, vor dem sich ganz rechts befindenden Schirm auf der
optischen Bank angeordnet sind.

Die einzelnen optischen Elemente (insbesondere Lampe und Linse) sollten beim Aufbau auf Parallelität
ausgerichtet werden. Dabei hilft es, die Lampe einzuschalten und dann das durch die Linse auf den
Schirm geworfene Lichtmuster zu beobachten. Wenn die Linse und die Lampe parallel zueinander sind,
sollte das Lichtmuster auf dem Schirm kreisförmig sein.

% Bild vom Versuchsaufbau
\begin{figure}[H]
    \centering
    \includegraphics[width=0.5\textwidth]{bilder/Versuchsaufbau1.jpg}           %// [x] TODO: Versuchsaufbau Bild anpassen
    \caption{optische Bank mit optischen Elementen}                             %// [x] TODO: Versuchsaufbau Bildunterschrift anpassen
    \label{Abb:Versuchsaufbau1}
\end{figure}

\subsection{Durchführung}
\label{sec:Durchfuehrung}
%//[ ] TODO: Durchführung
% Wie wurde der Versuch durchgeführt bzw. ausgewertet?
% auch wann und wo?

Vor dem tatsächlichen Versuch wurde zuerst die Brennweite der Linse grob abgeschätzt.
Dabei wurde die Linse über den Labortisch gehalten und so lange auf und ab bewegt, bis das
Abbild der Deckenbeleuchtung scharf erkennbar war. Der Abstand zwischen Linse und Abbildung
wurde dann mit dem Lineal abgemessen. Dieser Abstand entspricht annähernd der Brennweite
der Linse und beträgt ungefähr 9,6cm $\pm\:1\mathrm{cm}$. Dies kann man sagen, da der Abstand
zur Deckenleuchte im Verhältnis zum Abstand der Abbildung zur Linse hinreichend groß ist.

Nun kann mit dem eigentlichen Versuch begonnen werden, da jetzt $s>4f$ gewählt werden kann.
Nach Festlegung des ersten $s$ stellt man diesen Abstand zwischen Gegenstand und Schirm auf der
optischen Bank ein. Nun wird die Linse so verschoben, dass sich auf dem Schirm ein scharfes
Abbild des Gegenstandes zeigt. Die zweite Position, an der die Linse auch ein scharfes Bild erzeugt
befindet sich symmetrisch zum Mittelpunkt von $s$. 

Jedoch ist es schwierig, genau zu entscheiden, an welcher Stelle das Bild scharf ist. Deshalb wird jeweils für 
die Positionen ein ein Intervall bestimmt, dessen unteres Ende ein gerade noch nicht scharfes Bild und dessen oberes
Ende ein gerade nicht mehr scharfes Bild zeigt. Somit wurde ein Vertrauensintervall gewählt.

Die Messungen werden für verschiedene $s$ wiederholt, um eine genauere Bestimmung der Brennweite
zu erhalten. Die Messwerte werden im angehängten Laborprotokoll festgehalten.

\section{Messergebnisse und Auswertung}
%//[ ] TODO: Messung - evtl. nur Verweis auf Messergebnisse
% Eigentliche Messung!
% Wie groß sind die Messunsicherheiten („Messfehler“)?

Die Messwerte sind dem auf \url{https://eln.jku.at/} zugänglichen bzw. angehängten Laborprotokoll unter \ref{sec:Anhang} zu entnehmen.         %//[x] TODO: Protokoll Dateiname anpassen

%//[ ] TODO: Auswertung
% evtl. Formeln, etc.
% auf richtiges Runden der Werte achten
% evtl. auf Gleichungen Bezug nehmen

Die Größe $s$ wurde mit einer Unsicherheit von $\pm 0,5\mathrm{mm}$ gemessen. Übersetzt auf eine äquivalente
Normalverteilung ergibt sich eine Standardabweichung von $\sigma = 0,5\mathrm{mm} / \sqrt{3} \approx 0,03\mathrm{cm}$.
Im Gegensatz dazu wurden die Größen $e_1$ und $e_2$ mit einer Unsicherheit aus der Intervalllänge
ermittelt. Es wurde geschätzt, dass die Position, die ein tatsächlich scharfes Bild zeigt, mit einer
hohen Wahrscheinlichkeit (fast 100\%-ig sicher) innerhalb des Intervalls liegt. Somit kann das 
Intervall als $3\sigma$-Intervall ($n=3$) interpretiert werden. Dieses $\sigma\widehat{=}u_{e_i}$ ergibt sich aus:

\begin{equation}
    \label{eq:BesselAbstandUnsicherheit}
    u_{e_i} = \frac{e_{i_\mathrm{rechts}} - e_{i_\mathrm{links}}}{2\cdot n}
\end{equation}

Die Unsicherheit von $e$ entspricht der Kombination der Unsicherheiten von $e_1$ und $e_2$, siehe Gleichung \ref{eq:UnkorrUnsFortpfl}

Die Werte der Unsicherheiten sind folgender Tabelle zu entnehmen:

\begin{table}[H]
	\centering
	\begin{tabular}{c|c|c|c}
		$s$ /cm & $e_1$ /cm & $e_2$ /cm & $e$ /cm \\ \hline
		$45.0\pm0,03$ & $14,30\pm0,20$ & $30,85\pm0,12$ & $16,55\pm0,32$ \\
		$50.0\pm0,03$ & $13,10\pm0,17$ & $36,75\pm0,08$ & $23,65\pm0,25$ \\
		$55.0\pm0,03$ & $12,65\pm0,22$ & $42,35\pm0,05$ & $29,70\pm0,27$ \\
		$60.0\pm0,03$ & $12,30\pm0,13$ & $48,80\pm0,07$ & $36,10\pm0,20$ \\
		$65.0\pm0,03$ & $11,70\pm0,10$ & $54,00\pm0,07$ & $42,30\pm0,17$ \\
	\end{tabular}
    \caption{Ermittlung von $e$, Unsicherheiten}
\end{table}

Wenn man nun in die Gleichung \ref{eq:BesselBrennweite} die Werte für $s$ und $e$ eingesetzt, erhält man die Brennweite $f$.
Zusammen mit der Unsicherheit $u_f$ aus der Gleichung \ref{eq:BesselBrennwUnsFortpfl} ergibt sich folgende Tabelle:

\begin{table}[H]
    \centering
    \begin{tabular}{c|c|c}
        $s$ /cm & $f$ /cm & $u_f$ /cm \\ \hline
        $45.0\pm0,03$ & $9,73$ & $0,06$ \\
        $50.0\pm0,03$ & $9,70$ & $0,06$ \\
        $55.0\pm0,03$ & $9,74$ & $0,08$ \\
        $60.0\pm0,03$ & $9,57$ & $0,07$ \\
        $65.0\pm0,03$ & $9,37$ & $0,06$ \\
    \end{tabular}
    \caption{Brennweite $f$ und Unsicherheit $u_f$}
\end{table}

Mit $f$ und $u_f$ können nun der gewichtete Mittelwert $\bar{f_g}$ und die gewichtete Unsicherheit $u_{\bar{f_g}}$
nach den Gleichungen \ref{eq:Mittelwert} und \ref{eq:GewichtUnsicherheit} berechnet werden.

\begin{equation*}
    \bar{f_g} = 9,606\mathrm{cm}
\end{equation*}

\begin{equation*}
    u_{\bar{f_g}} = 0,028\mathrm{cm}
\end{equation*}


\section{Diskussion}
%//[ ] TODO: Diskussion
% Wie vergleicht sich meine Messung mit anderen Messungen/Theorien?
% Ist der Messwert sinnvoll? Stimmt die Größenordnung?
% Wo wurden Fehler gemacht? Was kann man verbessern?
% Gegebenenfalls rekursiv auswerten oder nachmessen!
% ursprüngliche Fragestellung diskutieren
% zB Standardabweichung diskutieren, berechnete Größen nennen

Der gewichtet Mittelwert der Brennweite $\bar{f_g} = 9,606\mathrm{cm}$ stimmt überein mit der grob abgeschätzten
Brennweite, die zu Beginn des Versuches ermittelt wurde. Die Unsicherheit $u_{\bar{f_g}} = 0,028\mathrm{cm}$ ist
dafür, dass nur bei fünf verschieden $s$ gemessen wurde, relativ klein.
Somit lässt sich das Bessel-Verfahren als geeignete Methode zur Bestimmung der Brennweite einer Sammellinse
bezeichnen.

Trotzdem hätte ein größerer Datensatz zu einer aussagekräftigeren Bestimmung der Brennweite geführt.
Auch bei der Bestimmung der Intervallsgrenzen $e_{i_\mathrm{links}}$ und $e_{i_\mathrm{rechts}}$ war
die subjektive Einschätzung der Schärfe des Bildes ein Faktor, der Einfluss auf die Ergebnisse hatte.
Es hätte auch besser darauf geachtet werden sollen, dass die Schärfe wirklich nur von einer
Experimentatorin beurteilt wird und nicht von einer zweiten Person beeinflusst wird.\\
Eine objektivere Methode, um die Schärfe des Bildes zu bestimmen oder eine genauere Definition von
Schärfe, hätte die Ergebnisse wahrscheinlich verbessert.

Beim Versuch wurde als größtes $s$ der Wert $65,0\mathrm{cm}$ gewählt. Dieser Wert ist relativ groß,
was sich beim beurteilen der Schärfe äußerte. Bei größeren $s$ ist das Bild mit der Linse bei $e_2$
sehr klein und es ist schwieriger, die Schärfe zu beurteilen. Eine Wahl von kleineren $s$ hätte
diese Problematik verringert.

In diesem Versuch wurde die Unsicherheit der Beurteilung der Schärfe durch Vertrauensintervalle
abgeschätzt. Eine alternative Methode wäre die Bestimmung der Unsicherheit durch die Wiederholung
der Messung gewesen. Es bleibt offen, welche Methode die besseren Ergebnisse liefert, da in diesem
Versuch nur eine Methode angewandt wurde.


\newpage

\section{Anhang}
\label{sec:Anhang}

\begin{figure}[H]
    \centering
    \includegraphics[width=1\textwidth]{bilder/Protokoll_Bilder/Protokoll1.png}
    \caption{Laborprotokoll Seite 1}
    \label{fig:Protokoll1}
\end{figure}

\newpage

\begin{figure}[H]
    \centering
    \includegraphics[width=1\textwidth]{bilder/Protokoll_Bilder/Protokoll2.png}
    \caption{Laborprotokoll Seite 2}
    \label{fig:Protokoll2}
\end{figure}

\newpage

\begin{figure}[H]
    \centering
    \includegraphics[width=1\textwidth]{bilder/Protokoll_Bilder/Protokoll3.png}
    \caption{Laborprotokoll Seite 3}
    \label{fig:Protokoll3}
\end{figure}

\newpage

\begin{figure}[H]
    \centering
    \includegraphics[width=1\textwidth]{bilder/Protokoll_Bilder/Protokoll4.png}
    \caption{Laborprotokoll Seite 4}
    \label{fig:Protokoll4}
\end{figure}

\end{document}
