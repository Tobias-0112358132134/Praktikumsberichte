\documentclass[a4paper,12pt]{article}
\usepackage[T1]{fontenc}
\usepackage[utf8]{inputenc}
\usepackage{graphicx}
\usepackage{amsmath}
\usepackage{amsfonts}
\usepackage{amssymb}
\usepackage{booktabs}
\usepackage{float}
\usepackage{geometry}
\usepackage[german]{babel}
\usepackage{enumitem}
\usepackage{multicol}

\setlength\parindent{0pt}

\geometry{a4paper, left=25mm, right=25mm, top=10mm, bottom=20mm}

\title{Versuchsbericht 1. Versuch: \\ \textbf{Würfeln}}
\author{Eva Brandstätter (k12406599)\\Tobias Mittermair (k12412801)\\Gruppe: Freitag Vormittag\\Betreuer: Gerald Gmachmeir}
\date{\today}

\begin{document}

\maketitle

\section*{Einleitung}
%//TODO: Einleitung
% Was soll gemessen werden? (Ziel/Motivation/Hypothese)


\section*{Grundlagen}
%//TODO: Grundlagen
% (kurz!) Was muss ich über die zu messende Größe wissen?

\section*{Versuchsbeschreibung}
\subsection*{Versuchsaufbau}
%//TODO: Versuchsaufbau
% Wie sieht der Versuchsaufbau aus? (Skizze, Anleitung, Geräte, …)

% \begin{figure}[H]
%     \centering
%     \includegraphics[width=0.5\textwidth]{bilder/Versuchsaufbau.jpg}
%     \caption{Versuchsaufbau}
% \end{figure}

\begin{itemize}[noitemsep,topsep=0pt,parsep=0pt,partopsep=0pt]
    \item 
\end{itemize}

\subsection*{Durchführung}
%//TODO: Durchführung
% Wie wurde der Versuch durchgeführt bzw. ausgewertet?
% als Fließtext

\section*{Messergebnisse und Auswertung}
%//TODO: Messung
% Eigentliche Messung!
% Wie groß sind die Messunsicherheiten („Messfehler“)?

%//TODO: Auswertung
% evtl. Formeln, etc.
% auf richtges Runden der Werte achten

% \begin{figure}[H]
%     \centering
%     \includegraphics[width=\textwidth]{bilder/Diagramm.png}
%     \caption{Histogramm der Messwerte (* ohne Ausreißer)}
% \end{figure}

\section*{Diskussion}
% Wie vergleicht sich meine Messung mit anderen Messungen/Theorien?
% Ist der Messwert sinnvoll? Stimmt die Größenordnung?
% Wo wurden Fehler gemacht? Was kann man verbessern?
% Gegebenenfalls rekursiv auswerten oder nachmessen!
% ursprüngliche Fragestellung diskutieren
% zB Standardabweichung diskutieren, berechnete Größen nennen



\end{document}
